\documentclass[
  a4paper,
  justified,
]{tufte-handout}

\usepackage{microtype}
\usepackage{amsmath, amssymb}
\usepackage[ngerman]{babel}
\usepackage{blindtext}

\usepackage{algorithm}
\usepackage{algorithmic}
\floatname{algorithm}{Algorithmus}
\renewcommand{\algorithmicrequire}{\textbf{Eingabe:}}
\renewcommand{\algorithmicensure}{\textbf{Ausgabe:}}

\usepackage{tikz}

\begin{document}

\section{Einleitung}%
\label{sec:einleitung}

\begin{marginfigure}[10ex]
  \centering
  \begin{tikzpicture}
    \coordinate (z)  at (0,0);
    \coordinate (1)  at ( 90:1);
    \coordinate (2)  at (162:1);
    \coordinate (3)  at (234:1);
    \coordinate (4)  at (306:1);
    \coordinate (5)  at ( 18:1);
    \coordinate (6)  at ( 90:1.5);
    \coordinate (7)  at (162:1.5);
    \coordinate (8)  at (234:1.5);
    \coordinate (9)  at (306:1.5);
    \coordinate (10) at ( 18:1.5);

    \foreach \i in {1,...,10} {
      \fill (\i) circle (1.5pt);
    }

    \draw (6) -- (7) -- (8) -- (9) -- (10) -- (6);
    \draw (1) -- (6);
    \draw (2) -- (7);
    \draw (3) -- (8);
    \draw (4) -- (9);
    \draw (5) -- (10);
    \draw (1) -- (3) -- (5) -- (2) -- (4) -- (1);
  \end{tikzpicture}
  \caption{Der Petersen-Graph}
\end{marginfigure}

\blindtext

\section{Mathematik}%
\label{sec:mathematik}

Ein bisschen Mathematik im Text ist z.\,B. $x^2 + y^2 = z^2$.
Es geht aber auch anders:
\begin{align*}
  {(x+y)}^n & = \sum\limits_{k=0}^{n} \binom{n}{k} x^k y^{n-k}
  & \int\limits_{-\infty}^{+\infty} e^{-x^2} dx & = \sqrt{\pi}
\end{align*}

\section{Pseudocode}%
\label{sec:pseudocode}

\begin{algorithm}
  \caption{Berechne $y = x^n$}
  \begin{algorithmic}
    \REQUIRE $n \geq 0 \vee x \neq 0$
    \ENSURE $y = x^n$
    \STATE $y \leftarrow 1$
    \IF{$n < 0$}
    \STATE $X \leftarrow 1 / x$
    \STATE $N \leftarrow -n$
    \ELSE
    \STATE $X \leftarrow x$
    \STATE $N \leftarrow n$
    \ENDIF
    \WHILE{$N \neq 0$}
    \IF{$N$ ist gerade}
    \STATE $X \leftarrow X \times X$
    \STATE $N \leftarrow N / 2$
    \ELSE[$N$ ist ungerade]
    \STATE $y \leftarrow y \times X$
    \STATE $N \leftarrow N - 1$
    \ENDIF
    \ENDWHILE
  \end{algorithmic}
\end{algorithm}

\end{document}
