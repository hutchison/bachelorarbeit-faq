\section{Einleitung}%
\label{sec:einleitung}

\blindtext

\section{Mathematik}%
\label{sec:mathematik}

Ein bisschen Mathematik im Text ist z.\,B. $x^2 + y^2 = z^2$.
Es geht aber auch anders:
\begin{align*}
  {(x+y)}^n & = \sum\limits_{k=0}^{n} \binom{n}{k} x^k y^{n-k}
  & \int\limits_{-\infty}^{+\infty} e^{-x^2} dx & = \sqrt{\pi}
\end{align*}

\section{Pseudocode}%
\label{sec:pseudocode}

\begin{algorithm}
  \caption{Berechne $y = x^n$}
  \begin{algorithmic}
    \REQUIRE $n \geq 0 \vee x \neq 0$
    \ENSURE $y = x^n$
    \STATE $y \leftarrow 1$
    \IF{$n < 0$}
    \STATE $X \leftarrow 1 / x$
    \STATE $N \leftarrow -n$
    \ELSE
    \STATE $X \leftarrow x$
    \STATE $N \leftarrow n$
    \ENDIF
    \WHILE{$N \neq 0$}
    \IF{$N$ ist gerade}
    \STATE $X \leftarrow X \times X$
    \STATE $N \leftarrow N / 2$
    \ELSE[$N$ ist ungerade]
    \STATE $y \leftarrow y \times X$
    \STATE $N \leftarrow N - 1$
    \ENDIF
    \ENDWHILE
  \end{algorithmic}
\end{algorithm}
