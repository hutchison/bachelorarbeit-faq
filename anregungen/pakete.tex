% _das_ Mathepaket schlechthin:
\usepackage[
  %% Nummerierung von Gleichungen links:
  leqno,
  %% Ausgabe von Gleichungen linksbündig:
  fleqn,
]{mathtools}
% und dazu noch ein paar Mathesymbole und so:
% (muss vor mathspec geladen werden)
\usepackage{amsmath, amssymb}

% chemische Formeln
% (muss auch vor mathspec geladen werden)
%\usepackage[version=4]{mhchem}

\usepackage{microtype}

\usepackage{ifxetex}

\ifxetex{}
  % Um auch schöne Schriftarten auswählen zu können:
  \usepackage[MnSymbol]{mathspec}

  % Wir wollen, dass alle unsere Schriften für TeX und einander angepasst sind:
  \defaultfontfeatures{
    Ligatures=TeX,
    Scale=MatchLowercase,
  }
  % Die Hauptschriftart:
  \setmainfont[]{Minion Pro}
  % Die Matheschriftart:
  \setmathfont(Digits,Latin,Greek)[
    Numbers={Lining, Proportional}
  ]{Minion Pro}
  \setmathrm{Minion Pro}
  % Die Schriftart für serifenlose Texte (z.B. Überschriften):
  %\setallsansfonts[
  %  BoldFont = {MyriadPro-Regular},
  %]{Myriad Pro}
  \setallsansfonts[
    BoldFont = {Gill Sans},
  ]{Gill Sans}
  % Und die Schriftart für nichtproportionale Texte:
  \setallmonofonts[]{Fira Mono}
  \newfontface\titlepagefont{Gill Sans}
\else
  \usepackage[utf8]{inputenc}
\fi

% Deutsche Sprache bei Silbentrennung und Datum:
\usepackage[ngerman]{babel}

% Besser als "Lorem Ipsum"
\usepackage{blindtext}

% St. Mary Road, liefert Symbole für theoretische Informatik (z.B. \lightning):
%\usepackage{stmaryrd}

% nutze die volle Seite als Satzspiegel:
\usepackage[
  % Randbreite sei 1cm (sonst ist sie 1in):
  %cm,
  % Kopf- und Fußzeile werden miteinbezogen:
  %headings
]{fullpage}

% verbesserte Tabellen
% bietet u.a. die Spaltenmöglichkeit 'm{width}' = zentrierte Spalte mit fester
% Breite
\usepackage{array}

% kann komplexe Linien in Tabellen ziehen:
%\usepackage{hhline}

% mehrseitige Tabellen:
%\usepackage{longtable}

% Tabellen mit gedehnten Spalten:
%\usepackage{tabularx}

% Pimpt enumerate auf (optionales Argument liefert Nummerierung):
\usepackage{enumerate}

% Kann descriptions auf die selbe Höhe bringen:
%\usepackage{enumitem}

% Liefert Hyperlinks (\hyperref, \url, \href}
\usepackage{hyperref}
\hypersetup{%
  colorlinks=false,
  linkcolor=black,
  urlcolor=blue,
}

%\usepackage{cleveref}
% Farben (wie bei TikZ):
\usepackage[dvipsnames]{xcolor}
\definecolor{mygray}{gray}{0.8}

% Ändert den Zeilenabstand:
%\usepackage[
%  % nur eine Möglichkeit auswählen:
%  singlespacing
%  %onehalfspacing
%  %doublespacing
%]{setspace}

% Schönere Tabellen
% dazu gibt's neue Kommandos:
% - \toprule[(Dicke)], \midrule[(Dicke)], \bottomrule[(Dicke)]
% - \addlinespace: Extrahöhe zwischen Zeilen
\usepackage{booktabs}

% TODOs:
% Hinzufügen mit \todo{Text} oder auch \todo[inline]{Text}
\usepackage[
  ngerman,
  textwidth=2cm,
  textsize=tiny,
  backgroundcolor=white,
  linecolor=black,
]{todonotes}

% Schöne numerische Zitierungen:
%\usepackage[square, numbers]{natbib}

% Verbessert den Satz von Abbildungsüberschriften:
\usepackage{caption}

% Ermöglicht durch \begin{linenumbers} Zeilennummern anzuzeigen:
%\usepackage{lineno}

% Ermöglicht Zugriff auf die letzte Seite (z.B. \pageref{LastPage}):
%\usepackage{lastpage}

% Quelltext schön setzen:
%\usepackage{listings}

% Logische Beweise:
%\usepackage{bussproofs}

% Unterstreichungen (\uline, \uuline, \sout: durchgestrichen, \uwave):
%\usepackage{ulem}

% Kann alle möglichen Maße ändern
% will man Querformat, dann:
%\usepackage[landscape]{geometry}

% bietet gestrichelte vert. Linien in Tabellen (':')
%\usepackage{arydshln}

% Quelltext schön setzen:
% (verlangt "xelatex -shell-escape"!)
%\usepackage{minted}

% Algorithmen und Pseudocode:
\usepackage{algorithm}
\usepackage{algorithmic}
\floatname{algorithm}{Algorithmus}
\renewcommand{\algorithmicrequire}{\textbf{Eingabe:}}
\renewcommand{\algorithmicensure}{\textbf{Ausgabe:}}

% Bilder einbinden:
\usepackage{graphicx}

% um in Tabellen eine Zelle über mehrere Zeilen laufen zu lassen:
%\usepackage{multirow}

% Verbessert fließende Objekte
%
%     \begin{figure}[H]
%     ...
%     \end{figure}
% wird’s auch wirklich _hier_ gesetzt
%\usepackage{float}

% SI-Einheiten mittels \si{}:
%\usepackage[mode=text]{siunitx}
%\sisetup{%
%  output-decimal-marker={,},
%  per-mode=fraction,
%  exponent-product=\cdot,
%}
%\DeclareSIUnit\cal{cal}
%\DeclareSIUnit\diopter{dpt}
%\DeclareSIUnit\fahrenheit{F}
%\DeclareSIUnit\molar{\textsc{m}}
%\DeclareSIUnit\pH{pH}
%\DeclareSIUnit\gewprozent{Gew\%}
%\DeclareSIUnit\poise{P}

% nette Brüche mittels \sfrac{}{}:
%\usepackage{xfrac}

% Coole Zeichnungen:
%\usepackage{tikz}
%\usetikzlibrary{%
%  backgrounds,
%  %mindmap,
%  %shapes.geometric,
%  %shapes.symbols,
%  %shapes.misc,
%  %shapes.multipart,
%  %positioning,
%  %fit,
%  calc,
%  arrows,
%  %automata,
%  %trees,
%  %decorations.pathreplacing,
%  %circuits.ee.IEC,
%  intersections,
%  through,
%}
%\usepackage{pgfplots}
%\pgfplotsset{compat=1.16}

% eigens gebaute Lochmarken:
%\usepackage{eso-pic}
%\AddToShipoutPicture*{
  %\put(\LenToUnit{0mm},\LenToUnit{228.5mm})
    %{\rule{\LenToUnit{20pt}}{\LenToUnit{0.5pt}}}
  %\put(\LenToUnit{0mm},\LenToUnit{68.5mm})
    %{\rule{\LenToUnit{20pt}}{\LenToUnit{0.5pt}}}
%}

%\usepackage{titlesec}
%\titleformat*{\paragraph}{\itshape\mdseries} % chktex 6
% \titleformat{\section}
%   {\sffamily}{\thesection}{1em}{}

% ein Eintrag in einer description-Liste wird in ganz normaler Schrift angezeigt (kein
% sans-serif, kein fett):
\renewcommand{\descriptionlabel}[1]{\hspace{\labelsep}#1}

% Verbessert den Umgang mit Abstracts:
\usepackage{abstract}
\addto\captionsngerman{\renewcommand{\abstractname}{Abstract}}

% für schlaue Zitate:
\usepackage{epigraph}
\setlength{\epigraphwidth}{0.42\textwidth}

% Definitionen und Sätze:
\usepackage{amsthm}
% Definitionen, Probleme werden alle mit einem Zeilenumbruch gesetzt, z.B.:
%   Definition 1 (Graph)
%   Ein Graph G = (V, E) ist …
\newtheoremstyle{bonny}% schottisch für „ansehnlich“
  {9pt}% measure of space to leave above the theorem. E.g.: 3pt
  {6pt}% measure of space to leave below the theorem. E.g.: 3pt
  {}% name of font to use in the body of the theorem
  {}% measure of space to indent
  {\bfseries}% name of head font
  {\smallskip}% punctuation between head and body
  {\newline}% space after theorem head; " " = normal interword space
  {}% Manually specify head

\theoremstyle{bonny}
\newtheorem{definition}{Definition}
\newtheorem{problem}{Problem}

% Beispiele, Sätze, Theoreme, Lemmata werden alle *ohne* Zeilenumbruch gesetzt, z.B.:
%   Satz 1  Für alle Graphen gilt …
\newtheoremstyle{sweet}%
  {9pt}% measure of space to leave above the theorem. E.g.: 3pt
  {6pt}% measure of space to leave below the theorem. E.g.: 3pt
  {}% name of font to use in the body of the theorem
  {}% measure of space to indent
  {\bfseries}% name of head font
  {}% punctuation between head and body
  {1em}% space after theorem head; " " = normal interword space
  {}% Manually specify head

\theoremstyle{sweet}
\newtheorem{beispiel}{Beispiel}
\newtheorem{satz}{Satz}
\newtheorem{theorem}{Theorem}
\newtheorem{lemma}{Lemma}
\newtheorem{folgerung}{Folgerung}

%% coole Kopf- und Fußzeilen:
%\usepackage{fancyhdr}
%% Seitenstil ist natürlich fancy:
%\pagestyle{fancy}
%% alle Felder löschen:
%\fancyhf{}
%% Veranstaltung:
%%\fancyhead[L]{}
%% Seriennummer:
%%\fancyhead[C]{}
%% Name und Matrikelnummer:
%%\fancyhead[R]{}
%%\fancyfoot[L]{}
%\fancyfoot[C]{\thepage}
%%\fancyfoot[C]{\thepage\,/\,\pageref{LastPage}}
%% Linie oben/unten:
%\renewcommand{\headrulewidth}{0.0pt}
%\renewcommand{\footrulewidth}{0.0pt}

% besondere Zeichen:
%\usepackage{pifont}
%\newcommand{\cmark}{\ding{51}}%
%\newcommand{\xmark}{\ding{55}}%
%\newcommand{\richtig}{\textcolor{ForestGreen}{\cmark}}
%\newcommand{\falsch}{\textcolor{BrickRed}{\xmark}}

\newcommand{\BigO}{\mathcal{O}}

% für eigene Notizen:
\newenvironment{notiz}{
  \color{Maroon}
  \paragraph*{Notiz}
}{
  \color{black}
}
